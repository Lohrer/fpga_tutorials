\section{Programming}
\label{sec:programming}

\section{Programming Flash (OPTIONAL)}
The FPGA's configuration is volatile, meaning it is lost when it loses power.
Flash devices on the other hand are non-volatile, so they will keep their stored files even
without power.
The Nexys4 board has a flash memory onboard that we can use to store the FPGA configuration file.
Using this is completely optional for labs, but for the final project presentations you will need
to have the configuration loaded in flash so that you don't waste time programming the board.
The first thing we have to do is tell Vivado to generate a bin file during the Generate
Bitstream step.

% FIGS

Now re-run Generate Bitstream and connect to the hardware server.

% FIGS

Add a memory device by right-clicking on the part number and then selecting Add Configuration
Memory Device...