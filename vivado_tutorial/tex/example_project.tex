\section{Creating an Example Project}
\label{sec:example_project}

This section will take you through how to set up a project, create a simple design, run a
simulation, create a top-level design, synthesize and implement your design, and finally
program the FPGA.

Open Vivado, and select File$\rightarrow$New Project or click on Create New Project in the
quick start screen.

% LOTS OF FIGS

Create two source files, gates2 and gates2\_tb.
You can skip adding any files during project creation in the future, but for now to show how
this is done we will do it here.
After creating gates2\_tb, set it to simulation only.
After you've created both you should have the following showing:

% FIGS

On the next screen you pick the FPGA that we are targeting, in the case of the Nexys 4 board it is
an Artix 7 with part number xc7a100tcsg324-1.
The Basys 3 has an Artix 7 FPGA with part number xc7a35tcpg236-1.
You can use the filter categories and search box to limit the selection to the desired part number.

% FIGS

Since you listed two files to create, after creating the project a window will pop-up asking to
define the inputs and outputs of the two files.
They should be exactly as shown below, gates2\_tb has no inputs or outputs, and gates2 has a and b
as inputs and a 6-bit bus z[5:0] as an output.

% FIGS

Once you press OK, you should see the following screen, other than some extra sections that are
removed so that the screenshots fit in this document easier.
On the left is your sources list, which have all your project files listed.
If you click on gates2, it will open it in the right section and you will see the auto-generated
code skeleton.
You should see your a, b, and z signals you created earlier as inputs and outputs to the module.

% FIGS

Edit the file so that is looks like the following picture.
For easy copy-and-pasting the code you have to add is given here.
Note that all the auto-generated comments were removed, feel free to leave or remove them as
you wish, or even add your own comments! It is a good idea to add your name and date at the top
of the file in a comment.

% FIGS

Since the testbench file, gates2\_tb was listed as simulation-only, it will show up under the
simulation sources section.
Click on it to open it; it will be blank since we didn't add any inputs or outputs!

% FIGS

Add the code required to make a testbench for gates2, shown in the next picture.
This time you'll have to type it out.
If you can't read any part of it let us know, but at least on a computer screen it should be
plenty legible.
You can copy gates2 as a guide and edit, or start from scratch.
The latter might be a good idea to get more familiar with the sections of VHDL code.
Once you are done run a behavioral simulation by clicking on Run Simulation and then Run
Behavioral simulation in the Flow Navigator to the left.

% FIGS

When the simulation runs, you will want to press zoom-fit so that you can see the entire
simulation.
However, by default the simulation runs for 1 μs which happened to be longer than needed in
this case.
All the action happened in the first 100 ns!
Zoom in to the first 100 ns by clicking the zoom in button and dragging the scrollbar.
Alternatively, you can click and hold the mouse to zoom in, out, fit, or to a range.
When you zoom in, it zooms at the location of the cursor.
You can click on the waveform to put the cursor at that location.

% FIGS

You can see from time 0-20 ns that the inputs a and b are both 0, so the output should be 0x15.
You can check the truth tables of all the gates to see that this is correct.
Then the rest of the truth table is simulated in the simulation.
If your simulation doesn't look correct, go back and check your code to make sure it is correct.
Once the simulation is correct, we can add the top-level file that will connect the inputs and
outputs to some physical things on the board like switches and LEDs.

% FIGS

Just like in the testbench, in the top-level we are going to port-map in the gates2 component.
We will simply wire up the I/O of gates2 to the switches and LEDs.
Type in the following code so that your gates2\_top looks like the following picture.

% FIGS

Now we need pin constraints.
These will map our top-level inputs and outputs, for this example sw[1:0] and ld[5:0] to the
physical pins on the FPGA so that they are connected to the external switches and LEDs.
We have preconfigured the constraints file for you, so all you have to do is download it from
Moodle and add it to your project.

% FIGS

Now that the constraints have been added, we can generate the .bit file that will be
programmed to the FPGA.
The steps that are required are to run Synthesis, Implementation, and then Generate Bitstream.
However, as a shortcut, you can click on Generate Bitstream and it will automatically run the
other steps if they need to be run.

% FIGS
Open target$\rightarrow$Auto connect
Program device$\rightarrow$xc7a100t\_0
% FIGS


