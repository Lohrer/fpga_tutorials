\documentclass[letterpaper]{article}

% Margins
\RequirePackage[left=1in,right=1in,top=0.5in,bottom=0.5in, includefoot, includehead,%
                head=14pt]{geometry}

% Footer
\RequirePackage{fancyhdr}
% Turn on the style
\pagestyle{fancy}
% Clear the header and footer
\fancyhf{}
% Set title in header
\fancyhead[C]{\large ECE/CSI 3710 and ECE/CSI 5008 Lab Syllabus}
% Set last update date on the left, page number on the right
\fancyfoot[R]{\small\thepage}
\fancyfoot[L]{\small Updated \today}

% Reduce spacing between items
\usepackage{enumitem}
%\setlist{noitemsep}
\setlist{itemsep=0em}

\begin{document}

\subsection*{How to submit a lab}
\begin{itemize}
    \item Demonstrate your working lab to a TA in your lab period
    \item Upload files to Moodle
    \begin{enumerate}[topsep=0em]
        \item All VHDL files including testbench
        \item Constraints file (XDC)
    \end{enumerate}
\end{itemize}

\subsection*{How to demonstrate a lab}
\begin{itemize}
    \item Complete the lab as specified in the lab handout
    \item Lab work is individual, you can get help from others but copying code is not allowed
    \item Ask a TA to demonstrate the lab
    \item Show that your solution works both in simulation and on the board
    \item The TA will ask a question to check your understanding of the lab
\end{itemize}

\subsection*{Lab grading}
\begin{itemize}
    \item Labs are due in your lab session, whichever you are assigned to
    \item If time allows labs will be graded when you demonstrate, otherwise they will be
          graded later from the code submitted to Moodle
    \item You must demonstrate your lab to a TA to get a grade
    \item Each lab is worth 20 points: 10 points for demonstration, 10 points for correct code
    \item Late submissions have a 25\% per week penalty. Anytime after your lab sessions ends
          counts as the next week.
\end{itemize}

\subsection*{Lab space guidelines}
\begin{itemize}
    \item You don't have to shut down the lab computers, just log out
    \item USB cables can be borrowed but are not left at the stations due to them going missing
\end{itemize}

\subsection*{Vivado tips}
\begin{itemize}
    \item Vivado can take some time to open, be patient
    \item Pay attention to critical warnings and errors
    \item Click on the blue links to take you you directly to the error
    \item Errors may not be caused exactly where they are reported, for example check the line
          above for things like a missing semicolon
\end{itemize}

\subsection*{General tips}
\begin{itemize}
    \item Save files on the H: drive or a flash drive so you have a copy when you leave lab
    \item Double-check your uploaded files match the lab's deliverables
    \item Don't get behind on labs,­ most of them build on each other
    \item If you have a question outside of lab, feel free to email the TAs.
          It is suggested to put all of them on the email to get a quicker response.
\end{itemize}

\end{document}